% Data sythesis using the Robinson/Freyer/Hindriks model - Overview
% Written by Christopher Thomas.

\chapter{Overview}
\label{sect-over}

This document describes a model used to simulate interacting neural
populations in the cortex and thalamus. The output of the model is mean
firing rates of these neural populations, with noise-excited oscillations
in the alpha band.

The original model was described and rigorously analyzed in Robinson 2002.
Freyer 2011 describes an extension to the model where the noise input is
modulated by the cortex population's output, providing a closer match with
the oscillation spectra in biological data. Hindriks 2023 describes an
extension to the model that adds multiple independent copies of the
Robinson and Freyer model with coupling between instances. This is used
to model co-oscillation of different brain regions.

The version of the model described in Hindriks 2023 is referred to in this
document as the ``extended Robinson model''.

Chapter \ref{sect-robinson-model} describes the details of the model.
Chapter \ref{sect-howto} gives a summary of how the model is used; for
a detailed treatment, consult the reference manual and sample code.
Chapter \ref{sect-robinson-math} provides a detailed mathematical analysis
of the model.

Reference details:

\begin{itemize}
%
\item P. A. Robinson, C. J. Rennie, and D. L. Rowe, \textit{Dynamics of
Large-Scale Brain Activity in Normal Arousal States and Epileptic Seizures},
Physical Review E, 65, 041924, April 2002
%
\item F. Freyer, J. A. Roberts, R. Becker, P. A. Robinson, P. Ritter, and
M. Breakspear, \textit{Biophysical Mechanisms of Multistability in
Resting-State Cortical Rhythms}, Journal of Neuroscience, 31,
pp 6353--6361, April 2011
%
\item R. Hindriks and P. K. B. Tewarie, \textit{Dissociation Between Phase
and Power Correlation Networks in the Human Brain is Driven by Co-Occurrent
Bursts}, Communications Biology, 6, 286, March 2023
%
\end{itemize}

%
% This is the end of the file.
