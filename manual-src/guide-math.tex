% Neuroloop utilities - Synthesis model guide - Robinson model - Analysis
% Written by Christopher Thomas.

\chapter{Robinson Model Mathematical Analysis}
\label{sect-robinson-math}

%
%
\section{Low-Pass Filter Delays}
\label{sect-robinson-math-lowpass}

Applying the Laplace transform to Equation \ref{eq-robinson-potential} shows
that the effect of $\alpha$ and $\beta$ is to apply a low-pass filter to the
weighted sum of input firing rates (a second-order exponential smoothing
filter with poles at $-\alpha$ and $-\beta$).

\begin{equation}
\left ( \frac{1}{\alpha \beta} \right ) s^2 V_a(s)
+ \left ( \frac{1}{\alpha} + \frac{1}{\beta} \right ) s V_a(s)
+ V_a(s) = \sum_b \nu_{ab} \Phi_b(s)
\end{equation}
%
\begin{equation}
s^2 V_a(s) + (\alpha + \beta) s V_a(s) + \alpha \beta V_a(s)
= \alpha \beta \sum_b \nu_{ab} \Phi_b(s)
\end{equation}
%
\begin{equation}
(s + \alpha) (s + \beta) V_a(s) = \alpha \beta \sum_b \nu_{ab} \Phi_b(s)
\end{equation}
%
\begin{equation}
\frac{V_a(s)}{\sum_b \nu_{ab} \Phi_b(s)}
= \frac{\alpha \beta}{(s + \alpha) (s + \beta)}
\label{eq-robinson-lowpass-albet}
\end{equation}

Applying the Laplace transform to Equation \ref{eq-robinson-gamma} shows
that the effect of $\gamma$ is to apply a low-pass filter to the firing rate
(a second-order exponential smoothing filter with both poles at $-\gamma$).

\begin{equation}
\left ( \frac{1}{\gamma^2} \right ) s^2 \Phi(s)
+ \left ( \frac{2}{\gamma} \right ) s \Phi(s) + \Phi(s) = Q(s)
\end{equation}
%
\begin{equation}
s^2 \Phi(s) + 2 \gamma s \Phi(s) + \gamma^2 \Phi(s) = \gamma^2 Q(s)
\end{equation}
%
\begin{equation}
(s + \gamma)^2 \Phi(s) = \gamma^2 Q(s)
\end{equation}
%
\begin{equation}
\frac{\Phi(s)}{Q(s)} = \frac{\gamma^2}{(s + \gamma)^2}
\label{eq-robinson-lowpass-gamma}
\end{equation}

The effect of both of these filters is to suppress high-frequency
oscillations (those above the filter corner frequencies) and to delay
low-frequency oscillations by an amount approximately equal to the filters'
time constants. These delays and corner frequencies are listed in Table
\ref{tab-robinson-lowpass} (using the parameter values from
\ref{tab-robinson-params}).

\tabdef{%
\begin{tabular}{cccc}\hline
\textbf{Parameter} & \textbf{Value} & \textbf{Corner} & \textbf{Delay} \\
\hline
$\alpha$ & 50 sec$^{-1}$ & 8 Hz & 20 ms \\
$\beta$ & 200 sec$^{-1}$ & 32 Hz & 5 ms \\
$\gamma$ & 100 sec$^{-1}$ & 16 Hz & 20 ms$^*$ \\
\hline
\multicolumn{4}{l}
{\footnotesize $^*$Each pole at $-\gamma$ introduces a 10~ms delay; there
are two such poles.}
\end{tabular}
}{Robinson model low-pass filter corners and low-frequency delays.}
{tab-robinson-lowpass}

%
%
\section{Oscillation Modes}
\label{sect-robinson-math-modes}

A simplified diagram of the extended Robinson model is shown in Figure
\ref{fig-robinson-loops}. This is intended to make it easy to identify the
feedback loops that may support oscillations. Blue arcs indicate positive
coefficients, and red arcs indicate negative (inhibitory) coefficients.
As an approximation, the activity of different excitatory neuron populations
within the cortex is assumed to be the same, combining $\nu_{ee}$ and
$\nu_{ee\_ext}$. Additionally, multiplicative portion of the noise is
treated as a contribution to the $\nu_{se}$ arc. The average contribution of
the $\chi \sigma_n \nu_n$ term is zero, but the magnitude of that term
compared to the magnitude of $\nu_{se}$ indicates whether or not
multiplicative noise significantly contributes to that arc.

\figdef{\includegraphics[height=2.5in]{figures/robinson-model-loops}}
{Simplified diagram of the extended Robinson model, showing feedback loops.}
{fig-robinson-loops}

A list of potential oscillation loops and their oscillation frequencies is
given in Table \ref{tab-robinson-loops}. For loops consisting entirely of
positive coefficients, or with two negative coefficients, the oscillation
period is the time needed to complete a single circuit. For loops with one
negative coefficient, the oscillation period is the time needed to complete
two circuits around the loop (in the same manner as a ring oscillator).
Harmonics of these oscillation frequencies are also supported.

The number of arc traversals needed for one oscillation period is noted.
Per above, this may reflect either one or two cycles around the loop. Each
arc traversal involves gain from arc coefficients (noted in the table),
small-signal gain from the $Q(V)$ transfer function (omitted from the table),
and delay from the $\alpha$ and $\beta$ filter components. Delay
contributions from the cortex excitatory population $\gamma$ filter
component and from the cortex-thalamus loop are noted in the table where
applicable.

\tabdef{%
\begin{tabular}{ccccccc}\hline
\textbf{Label} & \textbf{Arcs} & \textbf{Gamma} & \textbf{C-T Loop} &
\textbf{Period} & \textbf{Frequency} & \textbf{Coupling Gain} \\
\hline
%
EE & 1 & Y & -- & 45 ms & 22 Hz & $(\nu_{ee} + \nu_{ee\_ext})$ \\
ES & 2 & Y & Y & 150 ms & 6.7 Hz & $\nu_{se} \cdot \nu_{es}$ \\
%
ERSI & 4 & Y & Y & 200 ms & 5.0 Hz &
$\nu_{re} \cdot \nu_{sr} \cdot \nu_{is} \cdot \nu{ei}$ \\
%
\hline
%
II & 2 & -- & -- & 25 ms & 20 Hz & $\nu_{ii}$ \\
%
EI & 4 & Y & -- & 140 ms & 7 Hz & $2 \cdot \nu_{ie} \cdot \nu_{ei}$ \\
SR & 4 & -- & -- & 100 ms & 10 Hz & $2 \cdot \nu_{rs} \cdot \nu_{sr}$ \\
%
ESI & 6 & Y & Y & 350 ms & 2.9 Hz &
$2 \cdot \nu_{se} \cdot \nu_{is} \cdot \nu_{ei}$ \\
ERS & 6 & Y & Y & 350 ms & 2.9 Hz &
$2 \cdot \nu_{re} \cdot \nu_{sr} \cdot \nu_{es}$ \\
%
\hline
\end{tabular}
}{Extended Robinson model oscillation modes. Top modes: single-cycle
(non-inverting). Bottom modes: two-cycle (inverting). Activation function
gain is not shown.}
{tab-robinson-loops}

Resonant loops explicitly described in Section IV of Robinson 2002 are the
ones marked ES, ERS, and SR in Table \ref{tab-robinson-loops}. Considering
only those loops, the oscillation periods described in Section V of
Robinson 2002 are consistent with the estimated periods of the ERS and
SR loops.

The authors of Robinson 2002 were primarily concerned with noise-excited
oscillations, and so only evaluated oscillation loops that included the
specific nucleus. Evaluation was expressed in terms of the transfer
function from the noise signal (input) to the firing rate of cortex
excitatory neurons (output). Resonant oscillations were presumed to occur
at frequencies for which this transfer function diverged (producing
arbitrarily large output for finite input). This work instead considers
gain within a loop, with resonant oscillations corresponding to a loop
gain exceeding unity. As this does not explicitly consider noise excitation,
all of the loops described in Table \ref{tab-robinson-loops} may be
analyzed.

%
%
\section{Operating Points}
\label{sect-robinson-math-dc}

As was described in Robinson 2002, the dynamics of the extended Robinson
model can be analyzed by considering unchanging (DC) firing rates, and
evaluating the small-signal gain around these operating points. While this
was used to find the transfer function
$\frac{\phi_e(\omega)}{\phi_n(\omega)}$ in Robinson 2002, here it is used
to find the small-signal loop gain (to determine which oscillating modes
are dominant for given parameter values).

At a given operating point, the sensitivity of the gain of the dominant
loops to each of the $\nu_{ab}$ coefficients provides insight into which
network connections are most relevant for influencing dynamcis at that
operating point. $\nu_{ee\_ext}$ is particularly of interest, as this
may be used as a proxy for the sensitivity of network dynamics to changes
in the connectivity matrix between different excitatory cortex neuron
populations.

For time-independent operating point analysis, Equation
\ref{eq-robinson-potential} reduces to:

\begin{equation}
V_a = \sum_b \nu_{ab} \phi_b
\label{eq-robinson-dc-potential}
\end{equation}

Equation \ref{eq-robinson-sigmoid} (defining $Q(V)$) is unchanged, and
Equation \ref{eq-robinson-gamma} reduces to:

\begin{equation}
\phi_a = Q(V_a)
\label{eq-robinson-dc-nogamma}
\end{equation}

Combining Equations \ref{eq-robinson-dc-potential} and
\ref{eq-robinson-dc-nogamma} gives:

\begin{equation}
V_a = \sum_b \nu_{ab} Q(V_b)
\label{eq-robinson-dc-scalar}
\end{equation}
%
\begin{equation}
\vec{V} = \mathbf{N} Q(\vec{V})
\label{eq-robinson-dc-vector}
\end{equation}

In Equation \ref{eq-robinson-dc-vector}, vector $\vec{V}$ contains as its
elements all $V_a$, matrix $\mathbf{N}$ contains as its elements all
$\nu_{ab}$, and the activation function Q acts separately on each element
of $\vec{V}$.

The activation function given in \ref{eq-robinson-sigmoid-prime} can be
rewritten as:

\begin{equation}
Q(V) = \frac{Q_{max}}{1 +
e^{\left ( \frac{V_{th}}{\sigma'_{th}} \right )}
e^{- \left ( \frac{V}{\sigma'_{th}} \right )}
}
\label{eq-robinson-sigmoid-shuffle-1}
\end{equation}

For firing rates that are much less than $Q_{max}$ (such as those plotted
in Robinson 2002), the exponential term dominates, and the activation
function can be approximated as:

\begin{equation}
Q(V) \approx Q_{max} \, e^{- \left ( \frac{V_{th}}{\sigma'_{th}} \right )}
e^{\left ( \frac{V}{\sigma'_{th}} \right )}
\label{eq-robinson-sigmoid-approx-full}
\end{equation}
%
\begin{equation}
Q(V) \approx Q_0 \, e^{\left ( \frac{V}{\sigma'_{th}} \right )}
\label{eq-robinson-sigmoid-approx}
\end{equation}
%
\begin{equation}
Q_0 = Q_{max} \, e^{- \left ( \frac{V_{th}}{\sigma'_{th}} \right )}
\label{eq-robinson-sigmoid-q0}
\end{equation}

The operating point equation then becomes Equation
\ref{eq-robinson-dc-vector-approx}, where exponentiation is performed
separately for each element of $\vec{V}$:

\begin{equation}
\vec{V} \approx
\mathbf{N} \, Q_0 \, e^{\left ( \frac{\vec{V}}{\sigma'_{th}} \right )}
\label{eq-robinson-dc-vector-approx}
\end{equation}

This system of exponential equations may be solved numerically.

Alternatively, a system of linear equations may be obtained by assuming
that potentials are small compared to $\sigma'_{th}$.
\textbf{This is not a robust assumption;} it is used as an estimate only.
With this caveat kept in mind, a linear approximation of the exponential
function may be used (the first-order Taylor expansion), giving the following
(where $\vec{1}$ denotes a vector where all elements are unity):

\begin{equation}
\vec{V} \approx \mathbf{N} \, Q_0 \,
\left [ \vec{1} + \left ( \frac{1}{\sigma'_{th}} \right ) \vec{V} \right ]
\end{equation}
%
\begin{equation}
\left ( \frac{1}{Q_0} \right ) \vec{V} \approx \mathbf{N} \, \vec{1}
+ \left ( \frac{1}{\sigma'_{th}} \right ) \mathbf{N} \, \vec{V}
\end{equation}
%
\begin{equation}
\left [ \left ( \frac{1}{Q_0} \right ) \mathbf{I}
- \left ( \frac{1}{\sigma'_{th}} \right ) \mathbf{N}
\right ] \vec{V} \approx \mathbf{N} \, \vec{1}
\label{eq-robinson-dc-vector-linear}
\end{equation}

Equation \ref{eq-robinson-dc-vector-linear} has the form
$\mathbf{A} \vec{V} = \vec{b}$, and so may be solved as a set of linear
equations. The operating point estimated using Equation
\ref{eq-robinson-dc-vector-linear} \textbf{must} be examined to confirm
that $|V_a| \ll \sigma'_{th}$ for all $V_a$. If this constraint does not
hold, the estimate is not correct.

As a representative example, Table \ref{tab-robinson-operating-point} shows
the operating point corresponding to the model parameters given in Table
\ref{tab-robinson-params} and the coupling coefficients given in Table
\ref{tab-robinson-couplings}. A single population was considered (no
$\nu_{ee\_ext}$ coupling). All of the operating point values satisfy the
constraint for exponential approximation validity ($\phi \ll 250$), and
while the constraint for linear approximation is not satisfied
($V \ll 3.3$), they are sufficiently close to the desired domain that the
linear approximation result can be used as a seed value for obtaining the
exponential approximation result via gradient descent.

\tabdef{%
\begin{tabular}{c|cccc}
\hline
 & \textbf{Measured} & \textbf{Exponential} & \textbf{Linear} & \\
\hline
$\phi_e$ & 4.2 & 4.1 & 4.8 & /sec \\
$\phi_i$ & 4.2 & 4.1 & 4.8 & /sec \\
$\phi_s$ & 3.3 & 3.2 & 4.0 & /sec \\
$\phi_r$ & 5.3 & 5.3 & 5.6 & /sec \\
\hline
$V_e$ & 1.51 & 1.44 & 2.01 & mV \\
$V_i$ & 1.51 & 1.44 & 2.01 & mV \\
$V_s$ & 0.75 & 0.66 & 1.41 & mV \\
$V_r$ & 2.34 & 2.31 & 2.49 & mV \\
\hline
\end{tabular}
}{Measured and estimated operating points for a representative model.
``Measured'' values were averaged over a 30-second simulation window.
``Exponential'' and ``linear'' values were estimated using Equations
\ref{eq-robinson-dc-vector-approx} (using gradient descent) and
\ref{eq-robinson-dc-vector-linear} (as a system of linear equations),
respectively.}
{tab-robinson-operating-point}

%
%
\section{Small-Signal Gain and Dominant Oscillations}
\label{sect-robinson-math-gain}

The small-signal gain $G_{ab}$ of any given arc is the derivative of its
output firing rate with respect to its input firing rate. For oscillations
with frequencies much lower than the $\alpha$, $\beta$, and $\gamma$ filter
corner frequencies, this may be estimated by taking the derivative of the
DC operating point equations (rather than requiring an analysis of the full
system dynamics):

\begin{equation}
G_{ab} = \frac{d\phi_a}{d\phi_b}
\approx \frac{d}{d\phi_b} \left [ Q(V_a) \right ]
\end{equation}
%
\begin{equation}
G_{ab} \approx Q'(V_a) \frac{dV_a}{d\phi_b}
\end{equation}
%
\begin{equation}
G_{ab} \approx
Q'(V_a) \frac{d}{d\phi_b} \left [ \sum_c \nu_{ac} \phi_c \right ]
\end{equation}
%
\begin{equation}
G_{ab} \approx Q'(V_a) \nu_{ab}
\label{eq-robinson-gain-qprime}
\end{equation}

The sigmoid response function is defined in terms of the logistic function:

\begin{equation}
\left \{
\begin{array}{l}
Q(V) = Q_{max} L \left (\frac{V - V_{th}}{\sigma'_{th}} \right ) \\
\\
L(x) = \frac{1}{1 + e^{-x}} = \frac{e^x}{1 + e^x} \\
\\
\sigma'_{th} = \frac{\sqrt{3}}{\pi}\sigma_{th} \\
\end{array}
\right .
\label{eq-robinson-sigmoid-logistic}
\end{equation}

The derivative of the logistic function is:

\begin{equation}
L'(x) = L(x) \left ( 1 - L(x) \right ) = \frac{e^x}{(1 + e^x)^2}
\label{eq-robinson-logistic-derivative}
\end{equation}

This gives the derivative of the sigmoid response function:

\begin{equation}
Q'(V) = Q_{max} L' \left ( \frac{V - V_{th}}{\sigma'_{th}} \right )
\frac{d}{dV} \left [ \frac{V - V_{th}}{\sigma'_{th}} \right ]
\end{equation}
%
\begin{equation}
Q'(V) = \frac{Q_{max}}{\sigma'_{th}}
L' \left ( \frac{V - V_{th}}{\sigma'_{th}} \right )
\end{equation}
%
\begin{equation}
Q'(V) = \frac{Q_{max}}{\sigma'_{th}}
L \left ( \frac{V - V_{th}}{\sigma'_{th}} \right )
\left ( 1 - L \left ( \frac{V - V_{th}}{\sigma'_{th}} \right ) \right )
\end{equation}
%
\begin{equation}
Q'(V) = \frac{Q_{max}}{\sigma'_{th}}
L \left ( \frac{V - V_{th}}{\sigma'_{th}} \right )
\left ( 1 - \frac{Q_{max}}{Q_{max}}
L \left ( \frac{V - V_{th}}{\sigma'_{th}} \right ) \right )
\end{equation}
%
\begin{equation}
Q'(V) = \frac{1}{\sigma'_{th}} Q(V) \left ( 1 - \frac{Q(V)}{Q_{max}} \right )
\label{eq-robinson-sigmoid-derivative}
\end{equation}

Combining this with Equation \ref{eq-robinson-gain-qprime} gives:

\begin{equation}
G_{ab} \approx \frac{\nu_{ab}}{\sigma'_{th}}
Q(V_a) \left ( 1 - \frac{Q(V_a)}{Q_{max}} \right )
\end{equation}
%
\begin{equation}
G_{ab} \approx \frac{\nu_{ab}}{\sigma'_{th}}
\phi_a \left ( 1 - \frac{\phi_a}{Q_{max}} \right )
\label{eq-robinson-gain}
\end{equation}

This is Equation 10 from Robinson 2002.

The small-signal gain of a loop at low frequencies is the product of
$G_{ab}$ for each arc in the loop (as with the $S_d$, $S_i$, and $S_r$
values in section IV of Robinson 2002).
At frequencies that approach the low-pass filter cutoffs described in
Section \ref{sect-robinson-math-lowpass}, Equation
\ref{eq-robinson-lowpass-albet} must be applied to attenuate each edge, and
Equation \ref{eq-robinson-lowpass-gamma} must be applied to attenuate any
edge leaving the excitatory cortex population node.
The small-signal gains for the loops in Table \ref{tab-robinson-loops} are
shown in Table \ref{tab-robinson-loopgain}. These were evaluated at the
exponential approximation operating point from
Table \ref{tab-robinson-operating-point}, with low-pass filter attenuation
applied.

\tabdef{%
\begin{tabular}{ccccc} \hline
\textbf{Label} & \textbf{Frequency} & \textbf{Cycle Time} &
\textbf{Cycle Gain} & \textbf{Envelope Tau} \\ \hline
%
EE & 22 Hz & 45 ms & 0.14 & -23 ms \\
II & 20 Hz & 25 ms & -0.68 & -66 ms \\ \hline
%
EI & 7.1 Hz & 70 ms & -1.39 & \textbf{210 ms} \\
ES & 6.7 Hz & 150 ms & 0.81 & -690 ms \\
SR & 10.0 Hz & 50 ms & -0.09 & -20 ms \\ \hline
%
ESI & 2.9 Hz & 175 ms & -2.93 & \textbf{163 ms} \\
ERS & 2.9 Hz & 175 ms & -0.56 & -300 ms \\ \hline
%
ERSI & 5.0 Hz & 200 ms & 0.68 & -520 ms \\ \hline
%
\end{tabular}
}{Small-signal gains of extended Robinson model oscillation modes. Positive
gains correspond to single-cycle loops; negative gains to two-cycle loops.
Gains with absolute values smaller than 1 correspond to damped loops; gains
with absolute values greater than 1 correspond to oscillation modes that
grow over time with time constant $\tau_{loop}$.}
{tab-robinson-loopgain}

The single-cycle delay and single-cycle gain in a loop together define a time
constant over which that loop's oscillation's envelope grows or decays,
per Equations \ref{eq-robinson-gaintau} and \ref{eq-robinson-envtau}.
An envelope time constant that is positive indicates that an oscillation
mode grows with time; the smaller the time constant, the faster the growth:

\begin{equation}
| G_{loop} | = e^{\left ( \frac{t_{loop}}{\tau_{env}} \right )}
\label{eq-robinson-gaintau}
\end{equation}
%
\begin{equation}
\tau_{env} = \frac{t_{loop}}{\ln | G_{loop} |}
\label{eq-robinson-envtau}
\end{equation}

The dominant oscillation modes are expected to be those with the shortest
positive envelope time constants. Only two such loops are present with the
parameters from Tables \ref{tab-robinson-params} and
\ref{tab-robinson-couplings}: The EI loop in the cortex (at 7 Hz) and the
ESI loop through the thalamus (at 3 Hz). The gain in the ESI loop may
fluctuate due to the $\chi \sigma_n \nu_n$ contribution to $\nu_{se}$ (shown
in Figure \ref{fig-robinson-loops}). A power spectrum from a simulation
using these parameters is shown in Figure \ref{fig-robinson-spectrum},
showing a clear peak at 8 Hz (near the predicted EI loop resonance).

\figdef{\includegraphics[height=3in]{plots/20240108/spectrum-hindriks}}
{Power spectrum from a Robinson model simulation.}
{fig-robinson-spectrum}

%
%
\section{Adjusting Gain by Tuning Coupling Coefficients}
\label{sect-robinson-math-tuning}
%
% FIXME - Convenience macro.
\newcommand{\deln}{\nabla_{\mathbf{N}}}

Tuning the behavior of the simulated system is done by adjusting the
internal coupling coefficients $\nu_{ab}$ (aggregated as matrix
$\mathbf{N}$). The goal is typically to cause specific oscillation modes
to become dominant. To facilitate this, the sensitivity of loop gain with
respect to the coupling coefficients may be evaluated by computing the
gradient of gain with respect to $\mathbf{N}$.

The gain across a given network edge may be expressed in terms of the
edge's coupling coefficient and the derivative of the activation function,
per Equations \ref{eq-robinson-gain-qprime} and
\ref{eq-robinson-sigmoid-derivative}:

\begin{equation}
G_{ab} \approx \nu_{ab} \, Q'(V_a)
\end{equation}
%
\begin{equation}
Q'(V_a) = \frac{1}{\sigma'_{th}} \phi_a
\left ( 1 - \frac{\phi_a}{Q_{max}} \right )
\end{equation}

Taking the gradient with respect to $\mathbf{N}$ gives:

\begin{equation}
\deln G_{ab} \approx Q'(V_a) \deln \nu_{ab} + \nu_{ab} \deln Q'(V_a)
\label{eq-robinson-gain-gradient}
\end{equation}

$\deln \nu_{ab}$ is a matrix whose elements are zero except for a single
element at row $a$ column $b$, which is unity. The gradient of $Q'(V_a)$ is:

\begin{equation}
\deln Q'(V_a) = \deln \left [
\frac{1}{\sigma'_{th}} \phi_a \left ( 1 - \frac{\phi_a}{Q_{max}} \right )
\right ]
\end{equation}
%
\begin{equation}
\deln Q'(V_a) =
\frac{1}{\sigma'_{th}} \left (
\left ( 1 - \frac{\phi_a}{Q_{max}} \right ) \deln \phi_a
+ \phi_a \deln \left [ 1 - \frac{\phi_a}{Q_{max}} \right ]
\right )
\end{equation}
%
\begin{equation}
\deln Q'(V_a) = \frac{1}{\sigma'_{th}}
\left ( 1 - 2 \frac{\phi_a}{Q_{max}} \right ) \deln \phi_a
\label{eq-robinson-qprime-gradient}
\end{equation}

In principle, the gradient of the firing rate $\deln \phi_a$ may be
evaluated by taking the gradient with respect to $\mathbf{N}$ of Equations
\ref{eq-robinson-sigmoid-approx} and \ref{eq-robinson-dc-vector-approx}. In
practice, the firing rate gradients are evaluated numerically by finding an
operating point and making small perturbations to each $\nu_{ab}$.

The gradient of the gain along a path consisting of multiple edges can be
computed using the product rule. For two-, three-, and four-edge loops,
this gives:

\begin{equation}
\deln [ G_{ab} G_{ba} ] =
( \deln [ G_{ab} ] ) G_{ba} + G_{ab} ( \deln [ G_{ba} ] )
\label{eq-robinson-loop2-gain-gradient}
\end{equation}
%
\begin{equation}
\deln [ G_{ab} G_{bc} G_{ca} ] =
( \deln [ G_{ab} ] ) G_{bc} G_{ca} + G_{ab} ( \deln [ G_{bc} ] ) G_{ca}
+ G_{ab} G_{bc} ( \deln [ G_{ca} ] )
\label{eq-robinson-loop3-gain-gradient}
\end{equation}
%
\begin{equation}
\begin{array}{c}
\deln [ G_{ab} G_{bc} G_{cd} G_{da} ] =
( \deln [ G_{ab} ] ) G_{bc} G_{cd} G_{da}
+ G_{ab} ( \deln [ G_{bc} ] ) G_{cd} G_{da}
\\
+ G_{ab} G_{bc} ( \deln [ G_{cd} ] ) G_{da}
+ G_{ab} G_{bc} G_{cd} ( \deln [ G_{da} ] )
\end{array}
\label{eq-robinson-loop4-gain-gradient}
\end{equation}

While Equations \ref{eq-robinson-loop2-gain-gradient} through
\ref{eq-robinson-loop4-gain-gradient} can be expanded and written in terms
of $\nu_{ab}$, $\phi_{a}$, and $\deln \phi_{a}$, it is not particularly
useful to do so (since with or without such expansion the gradient will
have to be evaluated numerically).

Plots of the gradients of the loop gains of the EI and ESI loops (from
Table \ref{tab-robinson-loopgain}) are shown in Figure
\ref{fig-robinson-loopgrad}. The ``raw'' gradients are the change in loop
gain with respect to each of the $\nu_{ab}$ coupling weights (i.e.
$\deln G_{loop}$). The ``normalized'' gradients are divided by $G_{loop}$,
to show the \textit{relative} change in loop gain resulting from changes
to $\nu_{ab}$. In particular, this flips the sign of the gradient for
loop gains that are negative (with a positive normalized gradient indicating
that the loop gain is getting \textit{larger}, not more positive).

\figdef{%
\begin{tabular}{cc}
\includegraphics[height=2in]{plots/20240110/grad-loop-EI-raw} &
\includegraphics[height=2in]{plots/20240110/grad-loop-ESI-raw} \\
\includegraphics[height=2in]{plots/20240110/grad-loop-EI-norm} &
\includegraphics[height=2in]{plots/20240110/grad-loop-ESI-norm} \\
\end{tabular}
}{Gradient of loop gain with respect to $\nu_{ab}$, for the EI loop (left)
and the ESI loop (right). Top row: raw gradient. Bottom row: normalized
gradient (gradient divided by $G_{loop}$).}
{fig-robinson-loopgrad}

The gradients may be visually inspected to gain a qualitative understanding
of which $\nu_{ab}$ coupling weights have substantial impact on which loops.
In particular, sensitivity to the strength of the $\nu_{ee}$ coupling in this
analysis indicates sensitivity to the strength of the $\nu_{ee\_ext}$
coupling between populations, and to the weights in the population mixing
matrix.

Inspection of the gradient plots may also guide hand-tuning of coupling
weights. The main qualitative take-away is that while there are typically
a small number of $\nu_{ab}$ coupling weights that most strongly affect a
given loop, any given weight affects many loops, so weights must be jointly
optimized against all loop gains rather than optimized individually against
a single loop gain.

The gradients may also be used as the basis of gradient-guided searches of
parameter space. In practice, the gradient may not need to be explicitly
computed for this, since optimization library functions such as
\texttt{fsolve} may compute it numerically themselves by sampling the
loop gain.


\fixme{Show how to find regions of parameter space where desired loops are
dominant, including ones where the mixing matrix matters. Show where the
regions Robinson plotted are. Show where the region Hindriks used is.}

%
%
% This is the end of the file.
