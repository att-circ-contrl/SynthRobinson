% Data sythesis using the Robinson/Freyer/Hindriks model - Using the Model
% Written by Christopher Thomas.

\chapter{Using the Model}
\label{sect-howto}

\begin{sloppypar}
Sample code for running a simulation is found in
``\verb|code-example-synth/do_synth_robinson.m|'' in the ``\verb|examples|''
folder. Sample code for finding steady-state operating points, analyzing
loops, evaluating small-signal gain, and tuning internal couplings to change
small-signal gain is found in
``\verb|code-example-analysis/do_analyze_robinson.m|'' in the
``\verb|examples|'' folder.
\end{sloppypar}

The following sections provide a high-level description of how each of
these tasks are performed.

%
%
\section{Running a Simulation}
\label{sect-howto-run}

To run a simulation, do the following:

\begin{sloppypar}
\begin{itemize}
%
\item Get a model parameters structure and an internal couplings matrix.
The ones returned by \verb|synthRFH_getModelParamsHindriks| are a reasonable
starting point. See \verb|MODELPARAMS.txt| and Section
\ref{sect-robinson-model} for details.
%
\item Make a population coupling matrix, if you intend to simulate multiple
interacting populations.
%
\item Call \verb|synthRFH_ftWrapper_simulateNetwork| to run the simulation
if you want the output in Field Trip format, or
\verb|synthRFH_simulateNetwork| if you just want a matrix of firing rates.
%
\item If you need to convert firing rates back to cell potentials, you can
use \verb|synthRFH_getSigmoidInverse|.
%
\end{itemize}
\end{sloppypar}

%
%
\section{Finding a Steady-State Operating Point and Identifying Loops}
\label{sect-howto-oppoint}

To find a network's steady-state behavior, do the following:

\begin{sloppypar}
\begin{itemize}
%
\item Get model parameters and internal couplings per Section
\ref{sect-howto-run}.
%
\item Call \verb|synthRFH_estimateOperatingPointExponential| to estimate
steady-state conditions. Pass \verb|[]| as the starting point (it'll
bootstrap itself without trouble).
%
\item Call \verb|synthRFH_findLoops| to identify loops present. See
\verb|LOOPINFO.txt| for details about the structure array it returns.
%
\item Call \verb|synthRFH_getEdgeGains| to get the small-signal gain of
all internal connections.
%
\item Call \verb|synthRFH_addLoopGainInfo| to calculate the gain and rise
or fall time of each detected loop and add it to the loop information
structure array. Pass \verb|{}| as the gradient information.
%
\end{itemize}
\end{sloppypar}

%
%
\section{Adjusting Internal Couplings Automatically}
\label{sect-howto-tuning}

To automatically adjust internal couplings, do the following:

\begin{sloppypar}
\begin{itemize}
%
\item Decide on optimization goals. This is expressed as a Nx2 cell array
listing loop labels and optimization conditions (example:
\verb|{ 'ES', 'grow' }| to specify that loop \verb|ES| should have
oscillations that grow over time). Valid goals are \verb|'biggest'|
(fastest-growing), \verb|'grow'|, \verb|'decay'|, and \verb|'dontcare'|.
Loops that aren't listed in the goals default to \verb|'dontcare'|.
%
\item Call \verb|synthRFH_optimizeCouplings| to try to perturb the
internal couplings to give the desired behavior. Typical values of
optimization parameters are \verb|taulimit = 1.0|, \verb|bestfactor = 1.2|,
and \verb|maxprobes = NaN|.
%
\item Check the optimization error value that it gives you. If the error
value is 0.1 or larger, it couldn't satisfy your request (optimization did
not converge).
%
\end{itemize}
\end{sloppypar}

\textbf{NOTE - This only gives reasonable output for small changes.} Large
adjustments of behavior tend to result in networks with unrealistic coupling
parameters, and large adjustments of coupling parameters tend to result in
networks with unrealistic behavior (all activity suppressed or locked into
one oscillation mode).

%
%
\section{Getting Gradient Information}
\label{sect-howto-gradient}

The gradient of edge gain and loop gain with respect to internal coupling
weights can provide useful insight into how the model is behaving.

To compute gradients, do the following:

\begin{sloppypar}
\begin{itemize}
%
\item Find the model's steady-state operating point and loop gains, per
Section \ref{sect-howto-oppoint}.
%
\item Find the gradient of the operating point by calling
\verb|synthRFH_getOperatingPointGradient|,
 with \verb|zerohandling = 'nonzero'| and
\verb|couplingstep = 0.01|.
%
\item Find the gradient of edge gains by calling
\verb|synthRFH_getEdgeGainGradients| (supplying the firing rate gradients
found in the previous step).
%
\item Find the gradient of loop gains by calling
\verb|synthRFH_addLoopGainInfo|, passing the edge gain gradients found
in the previous step.
%
\end{itemize}
\end{sloppypar}

Gradient calculation is discussed in detail in Section
\ref{sect-robinson-math}. In principle, gradients may also give insight into
how the model might be hand-tuned to exhibit desired behavior; in practice,
numerical optimization is simpler. Gradient plots can give insight into
whether a desired optimization is \textit{possible} or not.

%
% This is the end of the file.
